 \stepcounter{lecture}
 \setcounter{lecture}{42}
 \sektion{Lecture 42}

 What we would've done next:
 \begin{enumerate}
   \item Generalized princepal ideal theorem: $R$ noetherian, $\p$ minimal prime over
   $(a_1,\dots, a_n$, then $ht(\p)\le n$.\\
   Corollary 1: $\spec R$ satisfies DCC (the length of a chain from $\p$ is bounded by
   $ht(\p)< \infty$ ($R$ noetherian, so $\p$ is finitely generated)).\\
   Corollary 2: $R$ local $\Rightarrow$ $\dim R< \infty$\\
   (Nagata): There exist ``bad noetherian rings'' which are infinite dimensional noetherian
   domains.

   \item $R$ noetherian of dimension $n$ $\Longrightarrow$ $\dim R[x] = n+1$.\\
   In general, $n+1\le \dim R[x] \le 2n+1$ (``Seidenberg bounds''), and these bounds are
   tight.
 \end{enumerate}

 \begin{definition}
   A ring $R$ is \emph{catenary} if given any two primes $\p\subsetneq \p'$, any two
   maximal prime chains from $\p$ to $\p'$ have the same length.
 \end{definition}
 Nagata showed that there are noetherian domains which are not catenary.

 \begin{definition}
   If $\p\in \spec R$, then $\dim \p:= \dim R/\p$.
 \end{definition}
 \begin{theorem}
   Any $k$-affine algebra $S$ is catenary (even if $S$ is not a domain). In fact, any
   saturated prime chain from $\p$ to $\p'$ has length $\dim \p - \dim \p'$. If $S$ is a
   domain, then all maximal ideals have the same height.
 \end{theorem}
 \begin{proof}
   Consider any chain $\p\subsetneq \p_1\subsetneq \cdots \subsetneq \p_r = \p'$. Then we
   get the chain
   \[
    S/\p \twoheadrightarrow S/\p_1 \twoheadrightarrow \cdots \twoheadrightarrow S/\p_r
    = S/\p'
   \]
   Here $\p_i/\p_{i-1}$ is height 1 in $S/\p_{i-1}$, so each arrow decreases the
   transcendence degree by exactly 1. Therefore, $tr.d._k S/\p' = tr.d._k S/\p -r$.
   \[
    r = tr.d._k S/\p - tr.d._k S/\p' = \dim S/\p - \dim S/\p' = \dim \p-\dim \p'.
   \]
   To get the last statement, take $\p=0$ and $\p'=\m$. Then we get that $ht(\m)=\dim S$.
 \end{proof}
 Note that the last statement fails in general.
 \begin{example}
   Take $S=k\times k[x_1,\dots, x_n]$. Then $ht(0\times k[x_1,\dots, x_n])=0$, but
   $ht\bigl(k\times (x_1,\dots, x_n)\bigr) = n$.
 \end{example}
 But that example is not connected.
 \begin{example}
   $S = k[x,y,z]/(xy,xz)$.
 \end{example}
 But this example is not a domain. In general, for any prime $\p$ in any ring $S$, we
 have
 \[
    ht(\p) + \dim \p \le \dim S.
 \]
 \begin{theorem}
   Let $S$ be an affine algebra, with $\Min S = \{\p_1,\dots, \p_r\}$. Then the following
   are equivalent.
   \begin{enumerate}
     \item $\dim \p_i$ are all equal.
     \item $ht(\p)+\dim \p =\dim S$ for all primes $\p\in \spec S$. In particular, if $S$
     is a domain, we get this condition.
   \end{enumerate}
 \end{theorem}
 \begin{proof}
   $(1\Rightarrow 2)$ $ht(\p)$ is the length of some saturated prime chain from $\p$ to
   some minimal prime $\p_i$. This length is $\dim \p_i - \dim \p = \dim S - \dim \p$ (by
   condition 1). Thus, we get $(2)$.

   $(2\Rightarrow 1)$ Apply (2) to the minimal prime $\p_i$ to get $\dim \p_i=\dim S$ for
   all $i$.
 \end{proof}
 We finish with a (non-affine) noetherian domain $S$ with maximal ideals of different
 heights. We need the following fact.\\
 \underline{Fact}: If $R$ is a ring with $a\in R$, then there is a canonical $R$-algebra
 isomorphism $R[x]/(ax-1) \cong R[a^{-1}]$, $x\leftrightarrow a^{-1}$.
 \begin{example}
   Let $\bigl(R,(\pi)\bigr)$ be a DVR with quotient field $K$. Let $S=R[x]$, and assume
   for now that we know that $\dim S=2$. Look at $\m_2=(\pi,x)$ and $\m_1=(\pi x-1)$.
   Note that $\m_1$ is maximal because $S/\m_1 = K$. It is easy to show that
   $ht(\m_1)=1$. However, $\m_2\supsetneq (x)\supsetneq (0)$, so $ht(\m_2)=2$.
 \end{example}
 Now let's come back to result I.1.1. The result we've just proven says that $ax-1\in
 U(R[x])$ if and only if $a\in \nil R$.
