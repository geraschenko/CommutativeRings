 \stepcounter{lecture}
 \setcounter{lecture}{32}
 \sektion{Lecture 32}

 Supplemental points:
 \begin{itemize}
   \item A valuation ring is the same thing as a local B\'ezout domain.
   \item A 1-dimensional valuation ring $(R,\m)$ with $\m$ principal is the same thing as a DVR.
   \item If $R$ is a valuation ring and $n\le \infty$, then $\dim R=n$ if and only if
   $|\spec R|=n+1$.
   \item If $k\subseteq K$ is a subfield, then $Val_k(K)$ is the \emph{Zariski space} (or
   \emph{Zariski Riemann Surface}) for $K/k$.
 \end{itemize}

 Composition of places: Let $(R,\m),(R',\m')\in Val(K)$, with $R'\subseteq R$ (so that
 $\m\subseteq \m'\subseteq R'\subseteq R$). Then we showed that $R'\m\in Val(R/\m)$, so
 it comes with a place $\sigma$ \anton{}, and we get a commutative diagram of places.
 \[\xymatrix{
  K \ar[r]^<>(.5)\phi_R \ar[dr]_{\phi'}^{R'} & (R,\m)_\infty \ar[d]^\sigma_{\frac{R'}{\m}}\\
  & (R',\m')_\infty
 }\]
 That is, $\phi'=\sigma\circ \phi$. We shoed earlier that $\dim R'=\dim (R'/\m)+\dim R$.
 We can phrase this as $\dim (\sigma\circ \phi)=\dim \sigma + \dim \phi$.

 Valuation rings on $K=k(x)$. Let's construct examples of $R\in Val_k(K)$ (points in
 the Zariski Riemann surface).
 \begin{example}
   Fix a monic irreducible $\pi(x)\in k[x]$, then define $R=k[x]_{(\pi)}$. $R$ is a DVR;
   it is called the ``$\pi$-adic valuation ring''. $\W =
   k[x]/(\pi) = k[\theta]$ (note that this is already a field), where $\theta = \bar x$.
   The $\pi$-adic place is $\phi:K\to \W_\infty$, with $\phi(f/g) = f(\theta)/g(\theta)$.
   Since $f/g$ can be assumed to be in lowest terms, we know when to send $f/g$ to
   infinity.

   In the special case where $\pi = x-a$, then $\theta = \bar x = a$, so the place is
   $f/g\mapsto f(a)/g(a)$.
 \end{example}
 \begin{example}
   Set $y = 1/x$, then $K=k(x)=k(y)$. Consider the $(y)$-adic place (or $1/x$ place) is
   the one with valuation ring $k[y]_(y)$. Let's figure out what this ring is in terms of
   $x$. If $r(x)$ is a non-zero rational function $\frac{a_0 x^n+\cdots+ a_n}{b_0x^m +\cdots
   +b_m} = \frac{x^n(\cdots)}{x^m(\cdots)} = y^{m-n} \frac{a_0+\cdots+ a_n y^n}{b_0 +\cdots
   +b_m y^m}$. Thus, $\phi(r(x)) =${\scriptsize $\begin{cases}
     0 & m> n\\
     \infty & m<n\\
     a_0/b_0 & m=n
   \end{cases}$}. So the valuation ring is $S=\{f/g|f,g\in k[x], g\neq 0, \deg f\le \deg
   g\}$, with uniformizer $y=1/x$.
 \end{example}
 \begin{theorem}
   $Val_k(K)$ is the exactly the set of rings described in the two examples above.
 \end{theorem}
 \begin{proof}
   Same as in the computation of all valuation rings of $\QQ$.
 \end{proof}
 If $k=\bar k$, then there are only linear irreducibles, so we get a valuation ring in
 $Val_k(K)$ for every point in $\mathbb{P}^1_k$. Thus the terminology ``Zariski Riemann
 surface''.

 \begin{definition}
   A valuation ring $(R,\m)$ has \emph{principal type} if $\m = (\pi)\neq 0$. We still
   call $\pi$ a uniformizer, but there is no noetherian hypothesis. Equivalently, we can
   say that $\m$ is non-zero and finitely generated (because $R$ is B\'ezout).
 \end{definition}
 These rings emulate DVRs. In dimension 1, these are exactly DVRs. Note that if $R$ is
 noetherian, Krull's principal ideal theorem doesn't apply, so $\m$ can have large height
 even through it is principal. We say that a place has principal type if the
 corresponding valuation ring does.

 \begin{claim}
   In the composition of places picture, if $\sigma$ (i.e.~$R'/\m$) has principal type,
   then so does $\phi'$ (i.e.~$R'$ has principal type).
 \end{claim}
 \begin{proof}
   Write $\m' = \pi R'+\m$, with $\pi\not\in \m$. Since $(\pi)\not\subseteq \m$, we
   must have $\m\subseteq (\pi)$. This implies that $\m'=(\pi)$, so $R'$ has principal
   type.
 \end{proof}
 \begin{example}
   \[\xymatrix{
    K = \QQ(x) \ar[r]^<>(.5)\phi_<>(.5){\QQ[x]_{(x)}} \ar[dr]_{\phi'} & \QQ_\infty \ar[d]^{\sigma_p}_{\ZZ_{(p)}}\\
    & (\FF_p)_\infty
   }\]
   We get that $\phi'$ has principal type because the maximal ideal of $\ZZ_{(p)}$ is
   principal. Let $R'$ be the valuation ring associated to $\phi'$. Then $\dim R'=2$ and
   $R'$ has principal type. $R' = \{f(x)/g(x)\big| x\nmid g(x), f(0)/g(0)\in
   \ZZ_{(p)}\}$. The uniformizer is $p$.
 \end{example}
