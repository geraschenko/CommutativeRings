 \stepcounter{lecture}
 \setcounter{lecture}{1}
 \sektion{Lecture 1}

 \subsektion{\S 1 Three theorems of McCoy}

  $R$ is always a commutative ring with unity $1_R$. $U(R)$ is the group of units of $R$.
  $\C(R)$ is the set of \emph{regular} elements of $R$, i.e.\ the set of
  non-zero-divisors. $\Z(R)$ is the set of zero-divisors, the complement of $\C(R)$.
  $(\{a_i\}) = (a_1,a_1,\dots)$ is the ideal generated by the set $\{a_i\}$; this is also
  written $\sum a_iR$. The principle ideal $(a)$ is $aR=Ra$. The notation $I\< R$ means
  that $I$ is an ideal in $R$.

 \begin{theorem}
   Let $A=R[x]$. Let $f=a_nx^n+\cdots +a_0\in A$. If there is a non-zero polynomial $g\in
   A$ such that $fg=0$, then there exists $r\in R\smallsetminus\{0\}$ such that $f\cdot
   r=0$.
 \end{theorem}
 \begin{proof}
   Choose $g$ to be of minimal degree, with leading coefficient $bx^d$. We may assume
   that  $d>0$. Then $f\cdot b\neq 0$, lest we contradict minimality of $g$. We must have
   $a_i g\neq 0$ for some $i$. To see this, assume that $a_i\cdot g=0$, then $a_ib=0$ for
   all $i$ and then $fb=0$. Now pick $j$ to be the largest integer such that $a_jg\neq
   0$. Then $0=fg=(a_0 + a_1x + \cdots a_jx^j)g$, and looking at the leading coefficient,
   we get $a_jb=0$. So $\deg (a_jg)<d$. But then $f\cdot (a_jg)=0$, contradicting
   minimality of $g$.
 \end{proof}
 \begin{theorem}[Prime Avoidance] \label{lec01T:Prime}
   Let $I_1,\dots, I_n\< R$. Let $A\subset R$ be a subset which is closed
   under addition and multiplication. Assume that at least $n-2$ of the ideals are
   prime. If $A\subseteq I_1\cup \cdots \cup I_n$, then $A\subseteq I_j$ for some $j$.
 \end{theorem}
 \begin{proof}
   Induct on $n$. If $n=1$, the result is trivial. The case $n=2$ is an easy argument: if
   $a_1\in A\smallsetminus I_1$ and $a_2\in A\smallsetminus I_2$, then $a_1+a_2\in
   A\smallsetminus (I_1\cup I_2)$.

   Now assume $n\ge 3$. We may assume that for each $j$, $A\not\subseteq I_1\cup \cdots
   \cup \hat I_j\cup \cdots I_n$.\footnote{The hat means omit $I_j$.} Fix an element
   $a_j\in A\smallsetminus (I_1\cup \cdots \cup \hat I_j\cup \cdots I_n)$. Then this
   $a_j$ must be contained in $I_j$ since $A\subseteq \bigcup I_j$. Since $n\ge 3$, one
   of the $I_j$ must be prime. We may assume that $I_1$ is prime. Define
   $x=a_1+a_2a_3\cdots a_n$, which is an element of $A$. Let's show that $x$ avoids
   \emph{all} of the $I_j$. If $x\in I_1$, then $a_2a_3\cdots a_n\in I_1$, which
   contradicts the fact that $a_i\not\in I_j$ for $i\neq j$ and that $I_1$ is prime. If
   $x\in I_j$ for $j\ge 2$. Then $a_1\in I_j$, which contradicts $a_i\not\in I_j$ for
   $i\neq j$.
 \end{proof}
 \begin{definition}
   An ideal $I\< R$ is called \emph{dense}\index{dense ideal} if $rI=0$ implies $r=0$.
   This is denoted $I\subseteq_d R$. This is the same as saying that ${}_RI$ is a
   faithful module over $R$.
 \end{definition}
 If $I$ is a principal ideal, say $Rb$, then $I$ is dense exactly when $b\in \C(R)$. The
 easiest case is when $R$ is a domain, in which case an ideal is dense exactly when it is
 non-zero.

 If $R$ is an integral domain, then by working over the quotient field, one can define
 the rank of a matrix with entries in $R$. But if $R$ is not a domain, rank becomes
 tricky. Let $\D_i(A)$ be the $i$-th \emph{determinantal ideal} in $R$, generated by all
 the determinants of $i\times i$ minors of $A$. We define $\D_0(A)=R$. If $i\ge
 \min\{n,m\}$, define $\D_i(A)=(0)$.

 Note that $\D_{i+1}(A)\supseteq \D_i(A)$ because you can expand by minors, so we have a
 chain
 \[
    R=\D_0(A)\supseteq \D_1(A)\supseteq \cdots \supseteq (0).
 \]
 \begin{definition}
   Over a non-zero ring $R$, the \emph{McCoy rank} (or just \emph{rank}) of $A$ to be
   the maximum $i$ such that $\D_i(A)$ is dense in $R$. The rank of $A$ is denoted
   $rk(A)$.
 \end{definition}
 If $R$ is an integral domain, then $rk(A)$ is just the usual rank. Note that over any
 ring, $rk(A)\le \min\{n,m\}$.

 If $rk(A)=0$, then $\D_1(A)$ fails to be dense, so there is some non-zero element $r$
 such that $rA=0$. That is, $r$ zero-divides all of the entries of $A$.

 If $A\in \MM_{n,n}(R)$, then $A$ has rank $n$ (full rank) if and only if $\det A$ is a
 regular element.

 \begin{exercise}
   Let $R=\ZZ/6\ZZ $, and let $A=diag(0,2,4)$, $diag(1,2,4)$, $diag(1,2,3)$, $diag(1,5,5)$
   ($3\times 3$ matrices). Compute the rank of $A$ in each case.
 \end{exercise}
 \begin{solution}\raisebox{-2\baselineskip}{
   $\begin{array}{c|cccc}
   A & \D_1(A) & \D_2(A) & \D_3(A) & \\ \hline
   diag(0,2,4) & (2) & (2) & (0) & 3\cdot (2)=0\text{, so }rk=0 \\
   diag(1,2,4) & R & (2) & (2) & 3\cdot (2)=0\text{, so }rk=1 \\
   diag(1,2,3) & R & R & (2) & 3\cdot (2)=0\text{, so }rk=2 \\
   diag(1,5,5) & R & R & R & \text{so }rk=3
  \end{array}$}
 \end{solution}
