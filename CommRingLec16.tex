 \stepcounter{lecture}
 \setcounter{lecture}{16}
 \sektion{Lecture 16}

 \noindent
 Exercise 68: You should assume that the characteristic of $k$ is not 2.\\
 Exercise 71: A non-Marot ring example.

 The easy form of Krull's intersection theorem is: If $(R,\m)$ is local noetherian and
 $M$ is finitely generated, then $\bigcap_{n\ge 0} \m^n M=0$.

 \begin{example}
   Counterexample if $R$ is not noetherian. Let $R=\QQ[x_1,x_2,\dots]/(x_{i+1}^2=x_i,x_1=0)$,
   with $\m=(x_1,x_2,x_3,\dots)$. Then $\m=\m^2=\m^3=\cdots$, so $\bigcap \m^n=\m\neq 0$.
 \end{example}
 \begin{example}
   Counterexample if $M$ is not finitely generated. Let $R=\ZZ_{(p)}$, which is local
   with maximal ideal $p\ZZ_{(p)}$. Take $M={}_{R}\QQ$, which is not finitely generated.
   Then we get that $\m M= pR\cdot M = p\QQ=\QQ=M$, so $\bigcap \m^n M=M\neq 0$.
 \end{example}

 Next we study a class of rings studied by Jean Marot (french guy). Recall that $\C(R)$
 is the set of non-zero-divisors. These elements are called regular.
 \begin{definition}
   An ideal $I$ is \emph{regular} if $I$ contains a regular element.
 \end{definition}
 \begin{definition}
   A ring $R$ is \emph{Marot}\index{Marot ring|idxbf} if every regular ideal can be
   generated by regular elements. (i.e.\ a regular ideal $I$ is generated by $I\cap
   \C(R)$.)
 \end{definition}
 Note that this class of rings includes integral domains (regular is the same as
 non-zero). Also, rings $R$ such that $\C(R)= U(R)$ are Marot (the only regular ideal is
 all of $R$). For example, zero-dimensional rings have this property. In particular,
 artinian rings are zero-dimensional.
 \begin{definition}
   $R$ has \emph{few zero divisors} if $\Z(R)$ is a \underline{finite} union of primes.
 \end{definition}
 \begin{lemma}
   Let $R$ be a ring with few zero divisors. Then for any $a\in R$ and any $b\in \C(R)$,
   there is some $r\in R$ such that $a+br\in \C(R)$. (i.e.\ every coset of $bR$ intersects
   $\C(R)$, provided $b$ is regular.)
 \end{lemma}
 \begin{proof}
   Write $\Z(R)=\bigcup_{i=1}^n \p_i$. We may assume there are no inclusions among the
   $\p_i$. After relabelling, we may assume that $a\in \p_i$ for $i\le k$ and $a\not\in
   \p_i$ for $i>k$. We may assume $k\neq 0$, lest $a$ be regular, in which case $r=0$
   works.

   If $\bigcap_{i=1}^k\p_i\smallsetminus \bigcup_{j=k+1}^n \p_j=\varnothing$, then by
   prime avoidance, there is some $j$ such that $\bigcap_i\p_i\subseteq \p_j$. Then since
   $\p_j$ is prime, we get $\p_i\subseteq \p_j$ for some $i$, contradicting the fact that
   there are no inclusions among the $\p$'s.

   Thus, we may choose $r\in \bigcap_{i=1}^k\p_i\smallsetminus \bigcup_{j=k+1}^n \p_j$,
   so $r\in \p_i$ exactly when $a\not\in \p_i$. Then we get $a+rb\in \C(R)$.
 \end{proof}
 \begin{theorem}
   Noetherian $\Rightarrow$ few zero divisors $\Rightarrow$ Marot.
 \end{theorem}
 \begin{proof}
   If $R$ is noetherian, then $\Z(R)$ is the union of the finitely many ``maximal
   primes'', $\bigcup_{\p\in \ass(R)^*} \p$.

   If $b\in I\cap \C(R)$, and $a\in I$, then by the Lemma, we get some $r_a$ so that
   $c_a=a+r_ab\in \C(R)\cap I$. Then $I$ is clearly generated by $b$, together with all
   the $c_a$.
 \end{proof}
 \[\xymatrix{
  & \text{Marot} \ar@{-}[d] \ar@{-}@/_/[ddl] \ar@{-}[dr]\\
  & \text{few 0-divs}\ar@{-}[d] \ar@{-}[dl] & \text{0-dim'l}\ar@{-}[dd]\\
  \text{domain} & \text{noetherian}\ar@{-}[dr]\\
  & & \text{artinian}
 }\]

 None of these implications is reversible. It doesn't take much thought to produce
 examples to demonstrate this.

 \begin{proposition}
   Suppose $R$ is Marot, and $I\< R$ is a proper regular ideal. Assume
   \[
     \text{for every }a,b\in \C(R),\qquad ab\in I\Rightarrow a\in I \text{ or } b\in I
     \tag{$\ast$}
   \]
   Then $I$ is prime.
 \end{proposition}
 \begin{proof}
   Assume $I$ is not prime, with $x,y\not\in I$, but $xy\in I$. Then $I+(x)\supsetneq I$
   is regular as well, so it is generated by regular elements. Those generators cannot
   all lie in $I$, so there is some generator $a\not\in I$. Similarly, there is some
   regular generator $b\in (I+(y))\smallsetminus I$. Then $ab\in (I+(x))(I+(y))\subseteq
   I$, contradicting ($\ast$).
 \end{proof}
 \begin{theorem}[E.~D.~Davis]
   A commutative ring $R$ has few zero divisors if and only if the total ring of
   quotients $Q(R)$ is a semi-local ring. In particular, if $R$ is noetherian, $Q(R)$ is
   semi-local.
 \end{theorem}
 \begin{proof}
   ($\Rightarrow$) Write $\Z(R)=\bigcup_{i=1}^n \p_i$, with no inclusions among the
   $\p_i$. Then $Q(R)$ is the localization at the complement of this set, which is the
   semi-localization of $R$ at this finite set of primes, so $Q(R)$ is semi-local.

   ($\Leftarrow$) Assume $Q(R)$ is semi-local, with $\Max(Q(R))=\{\m_1,\dots, \m_n\}$.
   Form the contractions $\p_i=\m_i\cap R$ (recall that $R\hookrightarrow Q(R)$). We
   claim that each $\p_i$ consists of zero-divisors; otherwise, $\p_i$ would contain a
   regular element, which would become a unit upon localization. Now we will show that
   $\p_1\cup \cdots\cup \p_n\subseteq \Z(R)$ is an equality. If $r\in \Z(R)$, then
   $r\cdot a=0$ for some $a\neq 0$, so $rQ(R)\neq Q(R)$, so $r\in \m_i$ for some $i$.
   Then $r\in \p_i$, as desired.
 \end{proof}
