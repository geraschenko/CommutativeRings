 \stepcounter{lecture}
 \setcounter{lecture}{3}
 \sektion{Lecture 3}

 \subsektion{\S 2. The Nilradical and Jacobson radical}

 The three spectra of a commutative ring $R$ are denoted $\spec(R) = \{\p\< R \text{
 prime}\}$, $\Max(R) = \{\m\< R\text{ maximal}\}$, and $\Min(R) = \{\p\< R \text{ minimal
 prime}\}$. $\p$ will always be a prime ideal.

 The \emph{nilradical} of $R$ is $\nil{R}=\{r\in R| r^n=0 \text{ for some }n\gg 0\}$.
 This is an ideal in $R$ (since $R$ is commutative). This is a special case of the
 ``radical formation''. If $I\< R$, then define $\sqrt{I}$ (sometimes denoted $\rad I$)
 to be the elements $r\in R$ so that $r^n\in I$ for some $n>0$.  In particular, $\nil R =
 \sqrt 0$.

 \begin{lemma}
   $\sqrt{I} = \bigcap_{\p \supseteq I}\p$
 \end{lemma}
 \begin{proof}
   The inclusion $\sqrt I \subseteq \bigcap_{\p \supseteq I}\p$ is clear. If $r\not\in
   \sqrt{I}$, then $\{1,r,r^2,\dots\}$ is disjoint from $\sqrt I$. Take an ideal $J$
   containing $I$ which is maximal with respect to not intersecting $\{1,r,r^2,\dots\}$.
   We wish to show that $J$ is prime, so assume $a,b\not\in J$ and $ab\in J$. Then there
   is some $r^n= j_1 + xa \in J+(a)$ and $r^m = j_2+yb\in J+(b)$ by maximality of $J$.
   But then
   \[
    r^{n+m} = j_1j_2 + xaj_2 + j_1yb + xyab \in J.
   \]
   Contradicting the construction of $J$.
 \end{proof}
 \begin{definition}
   $R$ is called \emph{reduced} if $\nil R=0$. An ideal $I$ is called \emph{reduced} (or
   \emph{radical}) if $I=\sqrt I$.
 \end{definition}

 \begin{definition}
   A prime $\p\supseteq I\< R$ is a \emph{minimal prime over $I$} if there there is no
   prime $\p'$ such that $I\subseteq \p'\subset \p$.
 \end{definition}
 Every $\p\supseteq I$ contains a minimal prime over $I$ (by Zorn's Lemma). In
 particular, $\sqrt I = \bigcap_{\p\supseteq I}\p = \bigcap_{\p'\text{ min'l over } I}
 \p'$.

 \begin{definition}
   The \emph{Jacobson radical} $\rad R$ is the intersection of all maximal ideals.
 \end{definition}
 Note that the maximal ideals of $R$ are in bijection with isomorphism classes of simple
 $R$-modules.\footnote{A \emph{simple} module is a non-zero module with no proper
 submodules.} How does the correspondence work? $\m\mapsto R/\m$, which is a simple
 module. On the other hand, given a simple module $S$, the annihilator of $S$ is a
 maximal ideal.\footnote{For noncommutative rings, many maximal ideals can correspond to
 the same isomorphism class of simple module.}

 Given this correspondence, one can conclude that $\rad R = \{r\in R| rS=0 \text{ for all
 simple modules } S\}$. In noncommutative theory, this is one definition of the Jacobson
 radical.

 \begin{lemma}[Key Property of $\rad(R)$]\label{lec03L:KeyRad}
   An ideal $I\< R$ is contained in $\rad (R)$ if and only if $1+I\subseteq U(R)$.
 \end{lemma}
 That is, $\rad R$ is the largest ideal $I$ such that $1+I\subseteq U(R)$.
 \begin{proof}
   Say $I\subseteq \rad R$, and $i\in I$. Then if $1+i$ is not a unit, it is in some
   maximal ideal $\m$. But $i\in \m$, so $1\in\m$. Contradiction.

   Conversely, assume $1+I\subseteq U(R)$ and that there is an $i\in I\smallsetminus\rad
   R$. Then there is some maximal ideal $\m$ that doesn't contain $i$. The ideal
   generated by $i$ and $\m$ is all of $R$, so we have $1=ri+m$ for $r\in R$ and $m\in
   \m$. Then $m=1-ri\in 1+I\subseteq U(R)$. Contradiction.
 \end{proof}

 \begin{lemma}
   $\nil R\subseteq \rad R$.
 \end{lemma}
 \begin{proof}[Proof \#1]
   Maximal ideals are prime, so $\bigcap \p\subseteq \bigcap \m$.
 \end{proof}
 \begin{proof}[Proof \#2]
   $1+\nil R\subseteq U(R)$ because $(1+r)^{-1}=1-r+r^2-\cdots \in R$ whenever $r$ is
   nilpotent. The result follows from Lemma \ref{lec03L:KeyRad}.
 \end{proof}
 \begin{proof}[Proof \#3]
   Any nilpotent element is in every maximal ideal.
 \end{proof}

 Next we discuss two important classes of rings.
 \begin{definition}
   $R$ is called \emph{Jacobson semisimple} if $\rad R=0$.
 \end{definition}
 \begin{definition}
   $R$ is called \emph{rad-nil} if $\rad R=\nil R$.
 \end{definition}
 Clearly J-semisimple implies rad-nil.
 \begin{example}
   Some J-semisimple rings: $\ZZ$, $k[x]$ when $k$ is a field, $\QQ[x,y]/(xy)$.
 \end{example}

 \begin{lemma}[Nakayama's Lemma 2.9]
   For $J\< R$, the following are equivalent:
   \begin{enumerate}
     \item $J\subseteq \rad(R)$
     \item Any finitely generated $R$-module $M$ satisfying $M=JM$ is zero.
     \item For any $R$-modules $N\subseteq M$ with $M/N$ finitely generated, $M=N+JM$
     implies $M=N$.
   \end{enumerate}
 \end{lemma}
 \begin{proof}
   $(1\Rightarrow 2)$ If $M\neq 0$, let $x_1$,\dots, $x_n$ is a minimal generating set
   for $M$ (with $n>0$). Since $JM = M$, we have $x_n= j_1 x_1+ \cdots + j_n
   x_n$ for $j_i\in J\subseteq \rad(R)$. But then $(1-j)x_n = j_1x_1+\cdots
   j_{n-1}x_{n-1}$, and $1-j\in U(R)$, so $x_n$ may be removed from the generating set,
   contradicting minimality.

   $(2\Rightarrow 3)$ Apply $(2)$ to $M/N$.

   $(3\Rightarrow 1)$ If $y\in J\smallsetminus \rad(R)$, then there is a maximal ideal
   $\m$ not containing $y$. But then $R = \m + JR$, so $(3)$ implies $\m=R$.
 \end{proof}

 \begin{corollary}[2.10]
   Let $J$ and $M$ be as above. Elements $x_1,\dots,x_n\in M$ generate $M$ if and only if
   their images $\bar x_1,\dots, \bar x_n$ generate $M/JM$.
 \end{corollary}
 \begin{proof}
   Take $N=Rx_1+\cdots + Rx_n$.
 \end{proof}

 \begin{definition}
   $R$ is \emph{local} if $|\Max R|=1$.
 \end{definition}
 We write ``$(R,\m)$ is local'', where $\m$ is the unique maximal ideal. Note that if
 $R$ is local then $R\neq 0$. Also note that we do not require $R$ to be Noetherian.
 \noindent Two major sources of local rings:
 \begin{enumerate}
   \item Take any maximal ideal $\m\< R$, and consider $R/\m^t$, where $t$ is a positive
   integer. The unique maximal ideal is $\m/\m^t$.

   \item If $\p\< R$ is a prime, then you can form the localization $R_\p$, whose unique
   maximal ideal is $\p R_\p$.
 \end{enumerate}
 Note that in a local ring $(R,\m)$, $U(R)=R\smallsetminus \m$. Also note that $R/\m$ is
 a field, called the \emph{residue field} of $R$.

 \begin{definition}
   $R$ is \emph{semi-local} if $|\Max R|<\infty$.
 \end{definition}
 \underline{Semi-localization}: Let $\p_1$, \dots, $\p_n\< R$ be primes. The complement of
 the union, $S=R\smallsetminus \bigcup \p_i$, is closed under multiplication, so we can
 localize. $R[S^{-1}] = R_S$ is called the \emph{semi-localization}
 \index{semi-localization} of $R$ at the $\p_i$.

 The result of semi-localization is always semi-local. To see this, recall that the ideals
 in $R_S$ are in bijection with ideals in $R$ contained in $\bigcup \p_i$. Assume $\p
 R_S$ is maximal in $R_S$, then $\p\subseteq \bigcup \p_i$. By prime avoidance (Theorem
 \ref{lec01T:Prime}), $\p$ must be in one of the $\p_i$, so the only maximal ideals are
 $\p_i R_S$.

 \begin{definition}
   For a finitely generated $R$-module $M$, define $\mu_R(M)$ to be the smallest number
   of elements that can generate $M$.
 \end{definition}
 This is not the same as the cardinality of a minimal set of generators. For example, 2
 and 3 are a minimal set of generators for $\ZZ$ over itself, but $\mu_\ZZ (\ZZ) =1$.

 \begin{theorem}
   Let $R$ be semi-local with maximal ideals $\m_1,\dots, \m_n$. Let $k_i = R/\m_i$. Then
   \[
     \mu_R(M) = \max \{\dim_{k_i} M/\m_i M\}
   \]
 \end{theorem}
 The proof is in the notes. \anton{find a short proof}
